


VARIE

 - Faisst+19 good description of the early galactic growth & panchromatic observations

1. from Bethermin+19

Understanding the early formation of the first massive galaxies
is an important goal of modern astrophysics. At z>4, most of
our constraints come from redshifted UV light, which probes the
unobscured star formation rate (SFR). Except for few very bright
objects (e.g., Walter et al. 2012; Riechers et al. 2013; Watson
et al. 2015; Capak et al. 2015; Strandet et al. 2017; Zavala et al.
2018; Jin et al. 2019; Casey et al. 2019), we have much less
information about dust-obscured star formation, i.e. the UV light
absorbed by dust and re-emitted in the far infrared. To accurately
measure the star formation history in the Universe, we need to
know both the obscured and unobscured parts (e.g., Madau &
Dickinson 2014; Maniyar et al. 2018).
With its unprecedented sensitivity, the Atacama large millimeter array (ALMA) is able to detect both the dust continuum
and the brightest far-infrared and submillimeter lines in "normal"
galaxies at z> 4. However, this remains a difficult task for blind
surveys. For instance, current deep field observations detect only
a few continuum sources at z>4 after tens of hours of observations (e.g., Dunlop et al. 2017; Aravena et al. 2016; Franco et al.
2018; Hatsukade et al. 2018). Targeted observations of known
sources from optical and near-infrared spectroscopic surveys are
usually more efficient. For instance, Capak et al. (2015) detected
four objects at z>5 using a few hours of observations.
The [CII] fine structure line at 158um is mainly emitted by
dense photodissociation regions, which are the outer layers of giant molecular clouds (Hollenbach & Tielens 1999; Stacey et al.
2010; Gullberg et al. 2015), although it can also trace the diffuse
(cold and warm) neutral medium (Wolfire et al. 2003), and to a
lesser degree the ionized medium (e.g., Cormier et al. 2012). It
is one of brightest galaxy lines across the electromagnetic spectrum. In addition, at z>4, it is conveniently redshifted to the
>850 µm atmospheric windows. This line has a variety of different scientific applications, since it can be used to probe the
interstellar medium (e.g., Zanella et al. 2018), the SFR (e.g.,
De Looze et al. 2014; Carniani et al. 2018a), the gas dynamics (e.g., De Breuck et al. 2014; Jones et al. 2020), or outflows
(e.g., Maiolino et al. 2012; Gallerani et al. 2018; Ginolfi et al.
2020). It has now been detected in ∼35 galaxies at z>4, but most
of them are magnified by lensing and/or starbursts and only one
third of them are normal star-forming systems (see compilation
in Lagache et al. 2018).
Over the past several years, numerous theoretical studies
have focused on the exact contribution of the various gas phases
(e.g., Olsen et al. 2017; Pallottini et al. 2019) and the effects of
metallicity (Vallini et al. 2015; Lagache et al. 2018), gas dynamics (Kohandel et al. 2019), and star-formation feedback (Katz
et al. 2017; Vallini et al. 2017; Ferrara et al. 2019) on the [CII]
emission, which is nowadays the most studied long-wavelength
line at z>4.
The rest-frame ∼160 µm dust continuum and the [CII] line
can be observed simultaneously by mation at z>4. The ALMA Large Program to INvestigate [CII] at
Early times (ALPINE) aims to build the first large sample with
a coherent selection process at z>4, increasing by an order of
magnitude the size of the pioneering Capak et al. (2015) sample.
Le Fèvre et al. (2019) describe the goals of the survey and Faisst
et al. (2019) present the sample selection and the properties of
galaxies in the sample, measured from ancillary data. In this paper, we present the processing of the ALPINE data from the raw
data to the catalogs and immediate scientific results such as the
basic dust and [CII] properties of the ALPINE targets together
with the number counts and redshift distribution of the serendipitous continuum detections.



2. form Fujimoto+20

On the CGM scale ( ∼ 1–10 kpc) at early cosmic times,
it has been revealed that star-forming galaxies are surrounded by extended Lyα line emission, namely the Lyα
halo, that is on average 10 times larger than their corresponding galaxy sizes in the rest-frame ultra-violet (UV)
wavelength (e.g., Momose et al. 2016; Wisotzki et al.
2016; Leclercq et al. 2017). While Lyα is fundamental to
probe the neutral hydrogen distribution around galaxies,
only metal lines allow us to probe the enriched gas distribution and hence constrain the efficiency of feedback
and star formation out to CGM scales.



3. From Angthopo+21

Galaxy formation and evolution represents one of the key important frontiers of astrophysics over the past decade. Owing to the complexity of physics concerning the transformation of gas into stars, a number of open questions remain.
In order to advance the field, a combination of two main
approaches are essential: (i) high quality surveys, most notably SDSS (York et al. 2000; Gunn et al. 2006), and (ii)
cosmological hydrodynamical simulations, such as EAGLE
(Schaye et al. 2015) or IllustrisTNG (Pillepich et al. 2018b;
Marinacci et al. 2018; Springel et al. 2018; Nelson et al. 2018;
Naiman et al. 2018). Both methods complement each other,
as observations help to constrain various parameters in simulations to reproduce the fundamental properties of galaxies,
while simulations enable the physical interpretation of the
observations.


4. From Kohandel+19

Answering the fundamental questions related to the formation,
build-up, and mass assembly of galaxies is one of the main goals of
modern astrophysics. The first stars and galaxies formed when the
diffuse baryonic gas in the Intergalactic Medium (IGM) was able to
collapse into the potential well of the dark matter halos in the early
universe. The ultraviolet (UV) radiation produced by these first
sources ionized the hydrogen atoms in the surrounding IGM. This
process, called cosmic reionization (Madau, Haardt & Rees 1999;
Gnedin 2000; Barkana & Loeb 2001), took about 1 billion years to
reach completion at z ∼ 6 (Fan et al. 2006; McGreer, Mesinger &
Fan 2011). After the formation of first sources, as time progressed,
those objects gradually evolved, merging with their neighbours and
accreting large quantities of gaseous fuel from a filamentary IGM.
Then, through a combination of galaxy–galaxy mergers, rapid star
formation, and secular evolution, the morphology of those galaxies
transformed into what is observed locally. Both observationally and
theoretically, understanding the details of the assembly process has
proven very challenging as the internal structure of these system
should be resolved

Among the FIR lines, the fine-structure transition 2P3/2 →
2P1/2 of singly ionised carbon at λ = 158μm is the brightest one,
accounting for 0.1 per cent to 1 per cent of the total FIR luminosity
(Stacey et al. 1991), making it as one of the most efficient coolants
of the ISM (Malhotra et al. 1997; Luhman et al. 1998, 2003).
Neutral carbon has a relatively low ionization potential (11.3 eV)
and its distinctive line transition ([C II]) is very easy to excite
(E/k ≈ 92 K). These properties are such that the line can arise from
nearly every phase in the ISM. It can emerge from diffuse H I clouds,
diffuse ionized gas, molecular gas, and from the photodissociation
regions (PDRs). So far, the [C II]158 μm line has been measured in
a rapidly increasing number of galaxies at z > 6 (e.g. Capak et al.
2015; Maiolino et al. 2015; Pentericci et al. 2016; Carniani et al.
2017; Jones et al. 2017; Matthee et al. 2017; Carniani et al. 2018a,b;
Smit et al. 2018).
Alongside observations, theoretical attempts have been made to
model the [C II] emission and interpret the observations at z > 6
(Vallini et al. 2013, 2015; Olsen et al. 2017;Pallottini et al. 2017a;
Katz et al. 2019) using numerical simulations of galaxies. So far,
the purpose of theoretical modellings was mostly to estimate the
total [C II] luminosity of galaxies at the EoR and understanding the
relative contribution from different ISM phases. These theoretical
works agree on the fact that most of the total [C II] luminosity
arises from the dense PDRs (Pallottini et al. 2017a) with a slight
dependence on galaxy mass (Olsen et al. 2017). Still no clear
consensus has been reached whether or not the local [C II] star
formation rate (SFR) relation that is observed locally (De Looze
et al. 2014) holds for z > 6 galaxies (cfr Carniani et al. 2018a).
For instance while Vallini et al. (2015) and Pallottini et al. (2017a)
show that a deviation is present, Katz et al. (2019) show that for
their suite of simulations at z ∼ 9, the local relation holds. The
[C II]–SFR relation is further analysed in different works (Ferrara
et al. in preparation; Pallottini et al. 2019), where it is connected to
galaxy evolutionary properties.


5. from Gong et al. 2012

Carbon is one of the most abundant elements in the
Universe and it is singly ionized (CII) at 11.26 eV,
an ionization energy that is less than that of the hydrogen. With a splitting of the fine-structure level at
91 K CII is easily excited resulting in a line emission
at 157.7 µm through the 2P3/2 →2 P1/2 transition. It
is now well established that this line provides a major
cooling mechanism for the neutral interstellar medium
(ISM) (Dalgarno & McCray 1972; Tielens & Hollenbach
1985; Wolfire et al. 1995; Lehner et al. 2004). It is
present in multiple phases of the ISM in the Galaxy
Wright et al. (1991) including the most diffuse regions
Bock et al. (1993) and the line emission has been detected from the photo dissociation regions (PDRs) of
star-forming galaxies (Boselli et al. 2002; De Looze et al.
2011; Nagamine et al. 2006; Stacey et al. 2010) and, in
some cases, even in z > 6 Sloan quasars (Walter et al.
2009).
The CII line is generally the brightest emission line
in star-forming galaxy spectra and contributes to about
0.1% to 1% of the total far-infrared (FIR) luminosity
(Crawford et al. 1985; Stacey et al. 1991). Since carbon
is naturally produced in stars, CII emission is then expected to be a good tracer of the gas distribution in
galaxies. Even if the angular resolution to resolve the
CII emission from individual galaxies is not available,
the brightness variations of the CII line intensity can
be used to map the underlying distribution of galaxies
and dark matter (Basu et al. 2004; Visbal & Loeb 2010;
Gong et al. 2011).


6. From Mitchell+21

t has long been recognised that galaxies must continually accrete diffuse gas from their surrounding environments in order to
explain the observed chemical abundances of stars (e.g., Larson
1972), and the relatively short gas depletion timescales of starforming galaxies (e.g., Bauermeister et al. 2010). This picture is
strongly supported by cosmological hydrodynamical simulations.
Simulations also demonstrate that feedback processes could plausibly affect diffuse gas accretion rates onto galaxies, both positively, by injecting metals into the circum-galactic medium, CGM,
and inter-galactic medium, IGM (facilitating radiative cooling), and
negatively, as galactic winds exert thermal over-pressure and kinetic ram pressure onto the surrounding gas (e.g., van de Voort
et al. 2011; Faucher-Giguere et al. 2011; Nelson et al. 2015; Correa `
et al. 2018). In addition, gas ejected from galaxies can in principle be later re-accreted, forming a distinct galactic wind recycling
contribution to galaxy growth (e.g., Oppenheimer & Dave 2008; ´
Oppenheimer et al. 2010; Ubler et al. 2014; Angl ¨ es-Alc ´ azar et al. ´
2017; van de Voort 2017; Mitchell et al. 2020a).


8. From Weiss+21
 
 
Here we use
a uniform sample of emission-line selected galaxies to
investigate the Lyα escape fraction. The difficulties associated with detecting and accurately measuring Lyα
from individual z ∼ 2 sources makes the identification
of trends in the Lyα escape fraction difficult to quantify. However, by co-adding (stacking) the smoothed,
low-resolution HETDEX spectra of 935 [O III] emitting
galaxies between 1.9 < z < 2.35, we have been able to
measure Lyα fluxes for galaxy samples where the Lyα
line is largely weak or undetectable. These data, when
compared to Hβ measurements from stacked 3D-HST
grism spectra, allowed for a determination of the escape
fraction of Lyα photons as a function of a wide range of
galaxy properties using a minimum of external assumptions.
Our data demonstrate that the fraction of Lyα photons escaping a galaxy depends greatly on the type of
galaxy being observed. In general, f
Lyα
esc and its dereddened counterpart, (f
Lyα
esc )*, are inversely correlated
with stellar mass, SFR, size, and internal E(B−V ), and
directly correlated with the strength of [O III] λ5007 relative to Hβ. However, not all of these correlations are
fundamental. In particular, the dependence of the Lyα
escape fraction on SFR is likely indirect, and only observed because SFR correlates with stellar mass. By disentangling these correlations, we determine that galactic stellar mass and dust reddening are the properties
with which f
Lyα
esc are most tightly linked. Future studies
aimed at detecting Lyα from complete samples of individual galaxies with Hα/Hβ derived extinction values
could yield more insight into which property (if any) is
the most fundamental in determining the Lyα escape
fraction.


9. From Tielens book

The key point always to keep in mind when studying the interstellar medium is
that the ISM is far from being in thermodynamic equilibrium. In thermodynamic
equilibrium, a medium is characterized by a single temperature, which describes [...] (final part of  intro chapter) 


10. From Tumlinson+17

Studies of outflow covering
fractions at z ∼ 1 reinforce a picture of outflows being roughly biconical, with little surface
area (∼ 5%) solely dedicated to inflow (Martin et al. 2012; Rubin et al. 2014).