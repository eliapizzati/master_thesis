%!TeX encoding = utf8
%!TeX spellcheck = it_IT
%!TeX source = ../thesis.tex




Investigating the complex environments of galaxies during the Epoch of Reionization (EoR, redshift $z>6$) is one of the most pressing research goals of modern astrophysics \citep{Barkana:2000fd, Dayal:2018hft}. As shown by cosmological simulations \citep{Murali_2002,Vogelsberger:2019ynw, somerville2015}, galaxies are already forming in the Epoch of Reionization, and they present different properties with respect to the ones seen in the local Universe. Unraveling how galaxies formed and evolved during these remote epochs (i.e., a few hundred million years after the Big Bang) is at the heart of our understanding of cosmic evolution.

Due to the non-linearity and large variety of physical phenomena occurring on a vast range of scales, modeling the galaxy formation and evolution process is very challenging. It is well established that galaxies originated from the growth of instabilities created by primordial matter fluctuations \citep{bertschinger98}. However, this alone is not sufficient to frame observations within a reliable theoretical model. Internal processes such as star formation \citep{moster2010constraints}, Active Galactic Nuclei (AGN) activity \citep{morganti2017archaeology}, feedback mechanisms \citep{fabian12}, as well as large-scale effects such as cosmic reionization \citep{mesinger_2016}, metal enrichment of the intergalactic medium (IGM) \citep{Aguirre:2001ay}, and mutual interaction between neighbouring galaxies \citep{dressler1980, delucia2007} need to be accounted for in order to build up a solid theoretical background that is able to keep pace with observational data.

In the last few years, astounding progress has been made in detecting galaxies that formed only a few hundreds of million years after the Big Bang. This has fueled a strong and widespread interest in detailed studies of early systems. As a result, the internal structure of high-redshift galaxies is now being routinely explored by radio-interferometers at Far-Infrared (FIR) and radio wavelengths \citep{Fevre:2019thf, capak2015, decarli2016alma}, as well as by large-scale surveys in the Near-Infrared (NIR) and optical bands \citep{refId0, Madau:1996yh}. 

In particular, the Hubble Space Telescope (HST) and the Very Large Telescopes (VLT) have revealed the size and the morphological properties of high-redshift galaxies by looking at the rest-frame ultraviolet (UV) wavelengths (which are redshifted into the NIR) \citep{oesch2009structure,Shibuya:2015qfa, bouwens2017z, kawamata2018size}. These studies have successfully characterized the evolution of the rest-frame galaxy UV luminosity functions (LFs), star formation, stellar buildup history, and size growth. For the very first time, we have a solid statistical characterization of galactic systems up to $z\approx 10$. 

At the same time, the advent of the Atacama Large Millimeter/submillimeter Array (ALMA) and the NOrthern Extended Millimeter Array (NOEMA) have opened a new window on the primordial Universe, exploring the obscured star formation and ISM line emission at rest-frame FIR wavelengths up to $z\approx7$ \citep{maiolino2015,pentericci2016, matthee2017, Hashimoto2018}. Combining the information coming from the dust continuum emission, as well as from some relevant FIR emission lines such as [CII] $158 \,\mu\mathrm{m}$, [O III] $88 \,\mu\mathrm{m}$, and CO from various rotational levels, we are rapidly improving our understanding of the small-scale, internal properties and assembly history of galaxies in the EoR, including their interstellar medium and relation to star formation \citep{capak2015, carniani2017}, gas dynamics \citep{smit:2018}, spatial offsets \citep{inoue2016, Laporte17, carniani2018}, dust and metal enrichment \citep{capak2015, Knudsen:2017, Laporte17, Tamura:2019}, molecular content \citep{dodorico:2018}, interstellar radiation field \citep{Stark15}, and outflows \citep{gallerani:2018, Fujimoto19, ginolfi:2019, herrera2021kiloparsec}.

The data coming from this plethora of observations can be compared to state-of-the-art, zoom-in cosmological simulations \citep{pallottini2017, Hopkins18, pallottini:2019}. In doing so, on one hand, we can benchmark the results of simulations and assess the physical assumptions on which they are based, and on the other hand, we can search in observational data for the features expected from theoretical considerations. One of the most important predictions made by cosmological simulations is known as the \textit{baryon cycle} \citep{peroux2020cosmic}. This process describes the history of baryons in the galactic environment, taking account of the complex interplay between star formation, stellar feedback, and accretion from the circumgalactic medium (CGM). Baryons are brought to great galactocentric distances ($1-10\,\mathrm{kpc}$ scales) by feedback mechanisms, but, at the same time, they are also accreted to the inner region of the galaxy due to the gravitational influence of the dark matter halo. Simulations clearly reveal how this process is responsible for driving galaxy growth \citep{tumlison}. Therefore, understanding the physics behind the competing processes of outflowing and inflowing matter is a key step to improve our understanding of cosmic evolution \citep{2005ARA&A..43..769V}.

Signs of galactic outflows were first found at low redshift a few decades ago \citep{chevalier_clegg:1985, Walter:2002vq}, but only very recently there has been strong evidence of the presence of outflows in high redshift galaxies. Outflows were seen in massive quasar-hosting galaxies \citep{cicone2015}, as well as in normal star-forming ones \citep{gallerani:2018, ginolfi:2019, Sugahara19, herrera2021kiloparsec}. These observations suggest a ubiquitous presence of outflows, thus supporting the claim that outflows steer galaxy evolution from the EoR to the present day \citep{veilleux2020cool}. 

However, despite the remarkable improvements made both from theoretical and observational perspectives, we are still lacking a deep and thorough description of the outflows driving mechanisms, and of their morphological, thermal, and chemical structure \citep{heckman2017galactic}. Both observations and detailed simulations suggest that outflows present a complex multi-phase structure \citep{murray2011, hopkins2014, muratov2015,pandya2021characterizing}. These phases often coexist together in the outflow, sharing the energy and mass budget of the outflowing gas. Moreover, hot gas can turn into a cold mode by emitting radiation and thus cooling down to a lower temperature via \textit{catastrophic radiative cooling}. This process can start when the threshold for H, He, and metal lines cooling is reached, and it rapidly lowers the temperature of the gas by several orders of magnitude. Different works highlight the role of this catastrophic cooling in regulating feedback mechanisms in super-star clusters \citep{Silich:2004,gray2019catastrophic}, and galaxies \citep{Wang:1995, sarkar:2015, Thompson16, Schneider:2018, Gronke&Oh:2020}.

Another important piece of the puzzle comes from recent observations of the circumgalactic medium (CGM) of $z\approx4-7$ galaxies. Several studies hint at the presence of singly ionized carbon surrounding these primeval galaxies \citep{Fujimoto19, Fujimoto:2020qzo, ginolfi:2019, herrera2021kiloparsec}. Using ALMA measurements of the \CII line emission, they show that carbon extends out to a distance that is significantly larger than the size of the galaxy itself, creating an \textit{extended halo} that dominates the properties of the galaxy's CGM. On average, the observed emission is $\gtrsim 5$ times spatially more extended than the UV/FIR stellar continuum. These findings pair with observations of extended Ly$\alpha$ emission at high redshift \citep{Wisotzki16,Wisotzki18, Kakuma19}, suggesting that halos are made by cold, neutral circumgalactic gas. Overall, these independent pieces of evidence for extended halos solidly confirm their presence. 

The discovery of extended halos around early galaxies poses a series of thorny theoretical questions, involving their formation mechanisms, evolution, and impact on the enshrouded galaxies as well as on the external regions where the intergalactic medium (IGM) resides. These problems become even more severe as we note that the most physically rich, zoom simulations \citep{pallottini2017b, Arata:2019} fail to reproduce the observed surface brightness distribution of the emitting material in extended halos.

This Thesis aims at taking on the problem of finding plausible mechanisms to explain the formation of these halos. In particular, we connect all the pieces of evidence here presented, focusing on the hypothesis that these halos result from the remnants of past -- or ongoing -- outflow activity. We explore this idea by using a semi-analytical model for an outflow that undergoes \textit{catastrophic cooling} in the inner region of the halo. Computing the abundance of singly ionized carbon and simulating the resulting \CII emission, we compare it directly with observational evidence coming from recent ALMA data \citep{Fujimoto19, Fujimoto:2020qzo}, and we conclude that outflows represent a promising answer to explain the origin of the observed \CII halos. 

The main features of the model here adopted are carefully described in \citet{Pizzati20}. In that article, the core structure from which this thesis work develops is laid down. The discussion here presented, however, also focuses on some important improvements on the original model, which are needed in order to obtain more reliable and physically-motivated results. A novel, thorough comparison with other observational and theoretical studies is presented here as well, and this leads us to draw more solid conclusions on the role of outflows in extended halos formation.


The thesis is structured as follows: 
\begin{itemize}
    \item In Chapter \ref{chap:intro}, we present a general introduction to the study of high redshift galaxies, briefly describing their formation and evolution stages, and focusing on the role of different physical mechanisms in shaping galaxies and affecting their physical properties.
    \item In Chapter \ref{chap:outflows}, we provide an insight into the physics of outflows and galactic feedback, inscribing the outflow model developed in the subsequent chapters inside the general theoretical framework on this subject. 
    \item Chapter \ref{chap:halos} is devoted to introducing \CII halos and the problem connected with their formation process. The observations of these halos are described in detail, and the tension with theoretical models is presented together with promising solutions. 
    \item Starting from Chapter \ref{chap:model}, we present our original work. We describe the main features of our model, focusing on the properties of the outflow and on the physics that give rise to the \CII transition line. We compute the final expected \CII emission and we study how it depends from some important physical parameters.
    \item In Chapter \ref{chap:results}, we compare the final predictions of our model with the observational data available to date. We find good correspondence between the extended \CII emission expected from our model and the measured one. Furthermore, we set up a Markov Chain Monte Carlo (MCMC) in order to explore the posterior distribution of the parameters of our model. In this way, we are able to infer the values of these parameters predicted by our model and to compare them with other observations and/or simulations. 
    \item Our conclusions are discussed in Chapter \ref{chap:conclusion}, along with some insights on future theoretical and observational efforts that will be able to further clarify the complex interplay between galaxies and their circum-galactic medium.
\end{itemize}

