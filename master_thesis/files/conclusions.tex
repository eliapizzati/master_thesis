

\vspace{13pt}


In this thesis work, we have argued that the recently discovered, very extended ($\approx 10-15\,\mathrm{kpc}$) \CII emitting halos around EoR galaxies are the result of supernova-driven cooling outflows. We have improved and extended the model presented in \citet{Pizzati20}, comparing its predictions with ALMA ALPINE observations of single systems at redshift $z\approx4-6$.

Having described the model and its results in detail, we are finally in the condition of giving some tentative answers to the compelling theoretical questions presented in section \ref{sec:theory_halos}:
\begin{itemize}
    \item Carbon is carried away from the center of galaxies (where it is produced by stellar nucleosynthesis) by hot modes of SF-driven winds. Energy and mass injections from SNe activity cause the gas to leave the galaxy with velocities in the range $300-500\,\kms$. Subsequently, the gas is gradually slowed down by the gravitational potential of the dark matter halo until it reaches a complete stop at radii $r\approx 10-15\,\mathrm{kpc}$. Thanks to this, we recover halos with finite sizes and masses around $M\approx 10^9\,\msun$.
    \item The success of the model largely relies on the fact that we follow precisely the catastrophic cooling of the outflow occurring within the central kpc. We find that cooling takes place for conditions (gas density $n\approx 1 \,\mathrm{cm}^{-3}$, temperature $T\approx 10^6\,\mathrm{K}$) consistent with the ones found by previous models and simulations \citep{Thompson16, Scannapieco:2017, gray2019catastrophic, danehkar_2021, fielding2021}. The gas cools very rapidly to $T \approx {\rm few} \times 100$ K, at the same time recombining. In this regime, the formation and survival of \CIIion ions are indeed possible. 
    \item We find that the outflow in our model is mostly neutral. This is because we assume that the escape fraction of ionizing radiation from the galaxy is negligible: thus, for the conditions of density and temperature in the outflow, hydrogen stays in the form of \HI, while carbon is collisionally ionized to \CIIion. The UV background radiation from other stars and quasars has indeed an effect on the gas, especially in the external low-density region. However, this does not prevent carbon to retain a significant abundance of \CIIion ions. 
    \item CMB suppression acts on the \CII emission by depleting the original \CII flux of factors between $2$ and $5$. Nevertheless, even accounting for this suppression effect we are able to recover the observed \CII luminosity ($L_\mathrm{CII}\approx 5-15\times10^8\,\mathrm{L}_\odot$).
    \item  Using our framework, we can infer crucial information on early galaxies, such as the outflow mass loading factor and the escape fraction of ionizing photons, that are hardly recovered from alternative methods at high redshifts. 
    \\By comparing our model predictions with observational data of individual ALPINE galaxies, we conclude that outflows are compatible with observations for values of the star formation rates and halo masses that lie in the range measured by pan-chromatic surveys.
    \\From this comparison, we infer values of the mass loading factor between $\eta\approx3$ and $\eta\approx10$, with inverse power-law dependences both on the stellar mass and on the star formation rate. Finally, we predict that a very low ionizing escape fraction from the parent galaxy is required ($\fesc \ll 1$). Values of $\fesc \gsim 0.2$, as those assumed by some reionization models, produce halo UV fields that are too intense for \CII to survive photoionization. 
\end{itemize} 

In brief, with our model, we can explain \CII halos as the result of cold neutral outflows from galaxies. Remarkably, although independent hydrodynamical simulations \citep{pallottini2017b, Arata:2019} have successfully matched both the dust and stellar continuum profiles deduced from observations \citep{Fujimoto19}, the same simulations could not reproduce the extended \CII line emission. This might be due to incomplete treatment of stellar feedback, or to numerical resolution issues related to the outflow catastrophic cooling. Our simple model, instead, can match the observed surface brightness. Hence, insight can be likely gained from a detailed comparison with simulations.

Alternatively, the failure of the simulations might indicate that the additional energy input required to transport the gas at such large distances could be provided by an AGN. This is suggested also by the values of the mass loading factor $\eta$ that we obtain from our analysis. These values are only marginally consistent with other measurements from starburst-driven outflows \citep{muratov2015, zhang2021empirical}, but they seem to be more compatible with the presence of AGN activity \citep[e.g.,][]{Fiore_2017}. This hypothesis must be tested via dedicated hydrodynamical simulations including radiative transfer.

The fact that the extended \CII halos surface brightness can be successfully fit by our model does not guarantee that outflows are the only possible explanation. Alternative interpretations, such as the presence of satellites, also need to be carefully explored. This is especially appropriate considering that, despite its success, the model presented here contains several limitations and hypotheses that will need to be further refined in the future:
\begin{itemize}
    \item The present one-dimensional treatment should be augmented with a full 3D numerical simulation of the outflow, also dropping the steady-state assumption made here. In this way, it may be possible to recreate a more realistic environment where non-spherical, truly multi-phase, outflows are launched by localized injection sites \citep{schneider2018production}. A similar picture is the one emerging from observations of the CGM local and high-redshift galaxies (chapter \ref{chap:halos}): outflows are often (bi-)conical-shaped, with different phases being characterized by different structures and morphologies. 
    \item A more realistic treatment of the circumgalactic/IGM environments is also necessary. Simulations show that accounting for an external CGM pressure might result in the formation of shocks in the outflowing gas \citep{samui:2008, Lochaas:2020, gray2019catastrophic}. Although we do not expect these shocks to dramatically affect the derived overall outflow structure, the detailed profile and extension of the \CII emitting region might turn out quantitatively different. This can be tested with less idealized, 3D models.
    \item Non-equilibrium cooling/recombination effects should also be considered when computing ionic abundances. In fact, several studies \citep[e.g.,][]{oppenheimer&schaye} have shown that assuming statistical equilibrium is generally not fair in very low-density environments illuminated by UV radiation.
\end{itemize}

%A first step towards an improvement of the model here considered is new parameters . 

Although some of these improvements might affect the quantitative conclusions of this work, it appears that so far outflows remain the best hypothesis to explain the puzzling nature of extended \CII halos. The role of outflows in extended halos formation could be tested in the future via dedicated ALMA observations. If so, we may be able to identify the smoking gun of the process by which the intergalactic medium was enriched with heavy elements during the EoR, taking another step forward in our understanding of the galaxy formation and evolution processes. 

%as witnessed by quasar absorption line experiments \citep{Dodorico13, Meyer19, Becker19}.
