\documentclass[12pt]{article}
\usepackage[utf8]{inputenc}

\title{Extended Thesis Abstract}
\author{Elia Pizzati, supervised by Andrea Ferrara and Andrea Pallottini}
\date{\vspace{-8pt}}



\usepackage[a4paper,top=2.3cm,bottom=2.3cm,left=2.8cm,right=2.8cm]{geometry}% heightrounded,bindingoffset=5mm




\usepackage{hyperref}

%\usepackage{hyperref}

\usepackage[round]{natbib}
\bibliographystyle{plainnat}


%\pagenumbering{gobble}

\begin{document}

\maketitle

%  OUTLINE (MAX 25.000 char ie 6 pages ca)

% - 
%
% - Observations of outflows and other stuff in the IGM
%
% - Importance of feedback
%
%
%
%



\paragraph{Introduction.}


This Thesis work develops from the model presented in \citet{Pizzati20}. In this ``Extended Abstract", I will briefly describe this model and the main results we obtained together with prof. Ferrara and the Cosmology Group at SNS. I will then mention the work we are currently undertaking and outline our plan for future efforts.

\paragraph{General context.}


Thanks to radio-interferometers such as ALMA and NOEMA, we can now investigate the internal structure of galaxies deep into the Epoch of Reionization (EoR, redshift $z>6$). 
Exploring the formation and the structure of galaxies during the EoR is crucial in order to gain insight into the latter stages of the galaxy evolution process. For this reason, there is a wide observational and theoretical effort directed at studying the complex interplay between galaxies and their surrounding circumgalactic/intergalactic medium \citep[CGM/IGM; for reviews see e.g.][]{Barkana:2000fd,Murali_2002,somerville2015,Dayal:2018hft,Vogelsberger:2019ynw}.

For this purpose, one of the most powerful methods is the spectroscopic analysis of the [CII] emission line. This line corresponds to the fine structure transition of singly ionized carbon $^2P_{3/2} \rightarrow \,^2P_{1/2}$ at 157.74 $\mu$m, and it is one of the brightest features in the FIR spectrum of high redshift galaxies \citep{lefevre:2019}. Thanks to this emission line, as well as others such as [O III] 88 $\mu$m line, CO from various rotational levels, and to the dust continuum emission, we are rapidly improving our understanding of the small-scale, internal properties and assembly history of galaxies in the EoR, including their interstellar medium and relation to star formation \citep{capak2015, carniani2017}, gas dynamics \citep{agertz:2015apj,pallottini2017,Hopkins18}, spatial offsets \citep{inoue2016, Laporte17, carniani2017, carniani2018}, dust and metal enrichment \citep{capak2015, Knudsen:2017, Laporte17, behrens2018, Tamura:2019}, molecular content \citep{dodorico:2018, vallini2018}, interstellar radiation field \citep{Stark15, pallottini:2019}, and outflows \citep{gallerani:2018, ginolfi:2019, herrera2021kiloparsec}.


\paragraph{The Extended Halos problem.} These observations are often leading us towards discoveries of new interesting features of these early systems that challenge the mainstream theoretical framework of galaxy formation. One of these findings involves the presence of gaseous \textit{extended halos} around high redshift galaxies.

The first evidence for the presence of these halos came from observations of the Ly$\alpha$ line with MUSE. By using 26 spectroscopically confirmed Ly$\alpha$-emitting galaxies at $3<z<6$, \citet{Wisotzki16} found that 21 objects showed the presence of a Ly$\alpha$ emission significantly (i.e. $\sim 5–15$ times) more extended than the central UV continuum. Similar works by \citet{Wisotzki18} and \citet{Kakuma19} identified very diffuse Ly$\alpha$ emission around high redshift galaxies, with a projected sky coverage approaching 100 percent.

With the advent of the ALMA interferometer, new evidence of extended halos developed thanks to spatially resolved observations of [CII] line emission. Early signs were found by \citet{cicone2015} in a massive $z\sim6$ quasar host: the [CII] emission detected by this study extended out up to $20–30\,\mathrm{kpc}$ from the galactic center, while the FIR emission did not exceed $15\,\mathrm{kpc}$. A compelling study by \citet{Fujimoto19}, then, combined 18 galaxies observed by ALMA at redshifts $z=5-7$, by applying a stacking technique directly in the uv-visibility plane. Again, the authors of this study found strong statistical evidence for a [CII] surface brightness which was around five times more extended than the HST stellar continuum and ALMA dust continuum. This corresponds to the presence of a Carbon halo extending out to approximately $10\,\mathrm{kpc}$ from the stacked galaxy center. Since the galaxies considered by \cite{Fujimoto19} have SFR between 10 and 100 $\mathrm{M}_\odot$, they are not expected to have a size greater than $1\,\mathrm{kpc}$: this means that most of the Carbon emission measured comes from the CGM of the galaxies. This finding pairs with observations of a $z = 3.042$ gravitationally lensed dusty star-forming galaxy \citep{Rybak20}, where $\sim50\,\%$ of [CII] emission was reported to arise outside the FIR-bright region of the galaxy.

More recently, detailed observations within the ALPINE program \citep{lefevre:2019, faisst:2019, bethermin:2019} proved the presence of extended halos on the individual scale. \citet{Fujimoto:2020qzo} measured the physical extent of [CII] line-emitting gas in 46 star-forming galaxies at $z=4-6$. This study found that, in the vast majority of cases, the size of the [CII] emitting region exceeds the size of the UV stellar continuum by factors of $\sim 2-3$, and concluded that $\sim 30\%$ of the sources have a [CII] halo extending over 10-kpc scales (note that the presence of selection biases could results in an underestimation of this value). Moreover, the authors studied the dependence of the size of this extended [CII] line structure on few galactic properties, finding a positive correlation with e.g. star-formation rate (SFR) and stellar mass. However, more detailed observations are needed to test whether CII halos are a universal feature in high-redshift star-forming galaxies, and to investigate the observed spread in halos' size and morphology.

Nevertheless, all of these independent studies pointing to the presence of halos around normal (and massive) star-forming galaxies at $z\sim6$ pose their existence on very solid grounds. It is therefore natural to search for a theoretical interpretation to explain how these halos formed, and how they interact both with the external environment and with the galaxies they host at their center. In particular, several challenging physical questions are raised by the discovery of these halos. First of all, the presence of Carbon (and presumably other heavy elements) far away from the center of the galaxy questions current models of galaxy evolution. In fact, at high redshift, the gas composing the CGM/IGM of galaxies is expected to have a very low metallicity, since metals are formed only by stars in the central regions of the galaxies. The presence of significant Carbon emission implies that Carbon has had to be either transported away from the galaxy or created by hidden star-formation activity in the external galactic environment. A second interesting question involves the ionization state of the gas: a bright [CII] line implies that Carbon remains in a singly ionized state even in the very low-density region of the CGM. This fact seems to be incompatible with the presence of an ionizing cosmic UV background produced by other galaxies and quasars: low-density, unshielded environments such as the Ly$\alpha$ forest \citep[e.g.][]{Dodorico13} usually host Carbon in higher ionization states.
Finally, [CII] emission in high redshift systems depends on many different physical processes \citep{pallottini2015, vallini2015, kohandel:2019}: for this reason, it is not clear what is the value of the Carbon mass that is needed to explain the observed surface brightness, and whether the presence of this mass could be physically motivated given the galaxy's properties. Such questions make clear that the origin, structure, and survival of [CII] haloes represent a formidable problem in galaxy evolution.


Moreover, relying on the results of cosmological simulations does not seem to be helpful in this particular context. In fact, even the most physically rich, zoom-in simulations \citep{pallottini2017b, Arata:2019} fail to reproduce the observed [CII] surface brightness. These independent studies agree in predicting a [CII] emission that is slightly more extended than the stellar continuum but drops very rapidly at distances considerably smaller than $10\,\mathrm{kpc}$. The resulting profiles are characterized by a value of the surface brightness which, in the external regions, is at least one order of magnitude lower than observed. This mismatch between theory and observation hints either at a non-optimal choice of the parameters governing these zoom-ins or at the presence of some undergoing physical process that has not been completely captured by numerical implementations. 

Different scenarios have been proposed to account for this unexpected [CII] emission: so far, the most convincing ones are the presence of satellite galaxies, galactic outflow activity, and the accretion of cold gas from the IGM. Cold accreting gas streams are considered to be one of the most important ways in which galaxies increase their gas content \citep[e.g.][]{Sancisi:2008wf}. However, their contribution in creating the observed Carbon emission is considered to be negligible from a theoretical standpoint, since these streams, coming directly from the IGM, are expected to have a very low content of metals.
The emission from a population of faint dwarf galaxy satellites is in principle a good candidate for solving the problem. However, while faint satellite galaxies are indeed seen in simulations, they do not provide a sufficient luminosity to account for the emission (Pallottini et al. 2019). Moreover, this hypothesis appears to be in contrast with observations: \citet{Fujimoto19} studied the ratio between [CII] emission and the total SFR surface density as a function of the galactocentric distance, proving that this ratio decreases when the external CGM is reached, and thus arguing against the presence of relevant hidden star-forming activity in the CGM region.


According to the outflow hypothesis, instead, the halos represent an incarnation of outflows driven by powerful episodes of star formation and/or active galactic nuclei (AGN) activity occurring in high-z galaxies. Indeed, AGN and starburst-driven outflows are expected to play a major role in galaxy formation, and their ubiquitous presence has been widely recognized \citep[see e.g.][]{Veilleux:2005ia}. Moreover, in the context of CII halo formation, recent independent ALMA observations by \citet{ginolfi:2019} and \citet{herrera2021kiloparsec} have linked the presence of extended spatial CII emission to the broad-wing feature in the CII line which is a characteristic sign of outflow activity. For these reasons, it appears that the outflow scenario is the most promising explanation for the problem of extended halos. In this Thesis work, we elaborate on this hypothesis by developing a physically motivated model for a galactic outflow and by comparing its resulting [CII] emission with the data coming from observations \citep{Fujimoto19, ginolfi:2019, Fujimoto:2020qzo}.



\paragraph{A brief introduction to galactic outflows.}

Galaxies cannot be considered as isolated systems evolving independently, with the rest of the universe acting as a passive background. Conversely, they are giant open ecosystems where baryons flow in and out in a complex interplay between galaxies, their surrounding CGM, and the external IGM. Galaxies accrete gas through inflowing streams, they grow in size and change their morphologies through mergers, and they expel material through feedback mechanisms. These latter mechanisms, in particular, are considered to be crucial to the process of galaxy evolution: they take the main form of powerful galactic outflows, both due to AGN and Supernovae (SNe), and they are routinely taken into account in theoretical, semianalytic and numerical simulations of galaxy formation and evolution \citep{Springel:2005nw, Vogelsberger:2014kha, Sijacki:2014yfa,schaye2015, Hopkins18}. These models show how galactic winds, ejecting gas from galaxies, deeply affect the star formation rate of galaxies \citep{Kormendy:2013dxa}, shape the stellar mass function and the mass-metallicity relation  \citep{dekelsilk, Finlator:2007mh}, and enrich the IGM with metals \citep{Aguirre:2001ay}. For this reason, quantifying their prevalence, dynamical importance, and observational signatures is a key problem in modern galaxy formation theory \citep[see e.g.][]{heckman2017galactic}.

Despite the great improvements made in recent years both from a theoretical and from an observational perspective \citep{veilleux2020cool}, we are still lacking a deep and thorough description of the outflows driving mechanisms, and of their morphological, thermal, and chemical structure. Both observations and detailed simulations point to the fact that outflows present a complex multi-phase structure composed of different modes \citep{murray2011, hopkins2014, muratov2015}. 

Hot modes ($T \approx 10^{6–7} \,\mathrm{K}$) are often fast and highly ionized, and they dominate the energy budget of the outflowing material. Hot outflowing gas is indeed observed in many systems thanks to its X-ray emission \citep{griffiths2000hot, strickland2009supernova}, and it is believed to be generated by the energy released by Supernovae into the ISM. Conversely, cool modes ($T \approx 10^{2-4} \,\mathrm{K}$) are neutral and slower, dominating the mass budget of the outflowing gas. The presence of these cold outflowing clouds in many rapidly star-forming galaxies and starbursts at all redshifts is well established by observations of blue-shifted absorption and emission. The nature of these clouds, however, represents a persisting puzzle in the theory of galactic outflows. The prevailing picture is that these clouds are simply formed by ISM gas that has been accelerated via ram pressure from the hot gas out of the host galaxy \citep[e.g.,][]{Veilleux:2005ia}. Nevertheless, a number of studies have challenged this explanation, claiming that ram pressure alone is not effective at accelerating the cool gas before it is shocked and shredded by hydrodynamical instabilities \citep{scannapieco2015launching, zhang2017entrainment}. Additional mechanisms that have been investigated include momentum deposition by supernovae, radiation pressure of starlight on dust grains, and cosmic rays \citep{Thompson:2015dynamics, quataert2021physics}. 

In an alternative model, \citet{Thompson16} suggested \citep[elaborating on earlier work by e.g.,][]{Wang:1995, Silich:2004} that the hot wind itself could turn into a cold mode by emitting radiation and thus cooling down to a lower temperature, in a process known as \textit{radiative cooling}. This cooling mechanism can take place only if the outflow is sufficiently mass-loaded, but it is indeed able to lower the temperature of the gas by several orders of magnitude in a timescale that can get significantly smaller than the outflow advection time. Different works have subsequently highlighted the role of this \textit{catastrophic cooling} in regulating feedback mechanisms in super-star clusters \citep{gray2019catastrophic}, and galaxies \citep{Schneider:2018, Gronke&Oh:2020}. 



\paragraph{Our model.} In order to investigate the role of outflows in forming Extended CII Halos, in this Thesis work we build a simple but reasonable model for an outflow. Our aim is to solve the hydrodynamical equations for the galactic wind and to derive physically-motivated density, velocity, and temperature radial profiles of the outflow as a function of the model parameters. These quantities will then form the basis for the prediction of [C II] luminosity profile.

We start by considering the classical model for a starburst-driven outflow proposed by Chevalier and Clegg \citep{chevalier_clegg:1985}. This model describes a spherically symmetric, hot, and steady wind that drives energy and mass - injected by stellar winds and supernovae - out from the center of the galaxy. Energy and mass are uniformly deposited by the central stellar cluster in an inner spherical region. Introducing the Star Formation Rate (SFR), the frequency of SNe in the galaxy ($\nu$), and the energy released by a single SN ($E_0$), we can write the mass and energy injection as $\dot{M}=\eta \,\mathrm{SFR}, \dot{E}=\alpha \nu E_0 \,\mathrm{SFR}$. $\eta$ and $\alpha$ are two dimensionless parameters that describe our ignorance about the efficiency with which mass and energy are injected into the ISM. 

Out of the deposit region, mass, momentum, and energy are conserved, and the wind expands adiabatically against the vacuum.
Solving the Euler equations in a time-independent fashion, we can find an analytical solution for the velocity of the gas $v$, its density $n$, and its temperature $T$. Looking at this latter profile, in particular, we realize that this outflow represents a hot mode, which can describe the formation of a halo but cannot be used for explaining the presence of CII outside the galaxy.
In fact, beyond the galactic radius, the temperature drops purely due to adiabatic cooling following the characteristic behavior $T\propto r^{-4/3}$. This results in a temperature which is greater than $10^4\,\mathrm{K}$ within the central 10 kpc for all models. At these temperatures, CII ions are still largely collisionally ionized to higher states, with the consequent suppression of the $158 \,\mu\mathrm{m}$ line emission. 


This suggests to modify the model including the effects of radiative losses. We follow \citet{Thompson16}, by considering an outflow that undergoes \textit{catastrophich cooling}, transitioning from an adiabatic hot phase in the inner region to a cool mode in the external CGM. In this way, provided that the mass loading factor $\eta$ is slightly greater than unity, we are able to obtain a uniform, spherically symmetric outflow with temperatures varying in the range $10^{2-4}\,\mathrm{K}$. These temperatures are indeed the ones required for having an ionization state of Carbon which is compatible with a significant presence of CII. We also account for the gravitational effect of the dark matter (DM) halo potential, because its influence is able to slow down the gas and subsequently increase its density in the outer regions of the outflow. In order to include gravity in our model, we parametrize the mass of the DM halo by introducing its circular velocity $v_c=\sqrt{GM(r)/r}$.

In practice, we consider for the inner galactic region the adiabatic model previously discussed, and we use the resulting physical conditions at the boundary between the galaxy and the CGM to integrate the Euler equations in the external region, accounting for gravity and radiative losses. This latter contribution is very complex, since it depends on the temperature, the density, and the chemical composition of the gas, as well as on the radiation fields acting on it. For this reason, we parametrize radiative cooling and heating via a net cooling function $\Lambda(T,n,Z, \xi_\mathrm{rad})$. Guided by observations and numerical simulations \citep{mannucci:2012, pallottini2017b}, we set solar metallicity ($Z=1$), and we use the approximation developed by \citet{gnedin2012cooling} to extract $\Lambda$ given the physical state of the gas and the photoionization rates of Hydrogen and Helium, as well as the Lyman-Werner band $\mathrm{H}_2$ photodestruction rate. These rates can be easily computed once the radiative flux acting on the gas is known: to this purpose, we consider the UV background (i.e., the sum of the ionizing radiation coming from all the other galaxy and quasars at a given redshift) in \citet{haardt2012radiative}, and we also take account of the radiation emitted by the central galaxy \citep{leitherer1999}. This latter quantity, however, depends on how much radiation is able to escape the dense and turbulent central regions of the ISM. As it is customary, we introduce an \textit{escape fraction} $f_\mathrm{esc}$ to account for this dependence. Unfortunately, $f_\mathrm{esc}$ is still largely unconstrained by observations \citep{Dayal:2018hft}. Therefore, to bracket all the possible physical configurations, we explore the results given by our model for $f_\mathrm{esc}=0$ and for $f_\mathrm{esc}=0.2$.

Using the photoionization rates for Hydrogen and Carbon (CI, CII, CIII), we can compute the ionization state of the outflow as a function of the galactocentric radius. We find that our choice of $f_\mathrm{esc}$ has sensible effects on the quantity of CII we finally obtain. In particular, if $f_\mathrm{esc}=0.2$, the outflow is strongly ionized, and Carbon is mainly found in the form of CIII and CIV. For a null escape fraction, instead, we are able to obtain a (approximately) neutral outflow with a significant amount of CII. This means that the presence of extended [CII] halos requires an ionizing escape fraction $f\ll 1$, because higher values of $f_\mathrm{esc}$ produce halo UV fields that are too intense for CII to survive photoionization. This raises the captivating possibility that outflows might be used to indirectly probe $f_{esc}$. 

Using $f_\mathrm{esc}=0$ as our fiducial model, we can finally compute the [CII] emission and directly compare our predictions for the surface brightness with the observational data from \citet{Fujimoto19}.  We find that our outflow model successfully matches the observed [CII] surface brightness if $\eta = 3.20 \pm 0.10$ and $v_c = 170 \pm 10 \mathrm{km/s}$. Given the properties of the sample considered in \citet{Fujimoto19}, these best-fit values of the parameters correspond to a galaxy mass of $M\sim 10^{11}\,\mathrm{M}_\odot$, and a mass outflow rate equal to $\dot{M}\sim 130 \,\mathrm{M}_\odot/\mathrm{yr}$. Such quantities are indeed compatible with the few observational constraints available to date \citep{gallerani:2018, herrera2021kiloparsec}. The mass outflow rate, however, is relevantly higher than the one inferred by \citet{gallerani:2018}, and the value of the mass loading factor - despite being marginally compatible with starburst driven outflows \citep[e.g.,][]{Heckman15} - is probably more typical of AGN \citep{Fiore_2017}. This could suggest the presence of (faint) AGN activity in the galaxies here considered. In summary, with the simplified model here presented, we have shown how extended halos might be used to set constraints on the mass loading factor and dark matter halo mass of early galaxies. 




\paragraph{Conclusion and current work.}


We presented a model \citep{Pizzati20} for a galactic outflow that is able to explain the formation of extended CII halos in high-redshift galaxies. The success of the model largely relies on the fact that we follow precisely the \textit{catastrophic cooling} of the outflow occurring within the central kpc. We find that cooling takes place for conditions (gas density $n  \approx 1\,\mathrm{cm}^{-3}$, temperature $T \approx 10^6 \,\mathrm{K}$) consistent with the ones found by previous models and simulations \citep{Thompson16, Scannapieco:2017, gray2019catastrophic}. The gas cools very rapidly to $T \approx \mathrm{few} \times 100\,\mathrm{K}$, at the same time recombining. In this regime, the formation and survival of C II ions is guaranteed. CII ions are transported by the neutral outflow at velocities of $300-500 \mathrm{km/s}$. In brief, [CII] halos, according to our model, are the result of cold neutral outflows from galaxies.

We applied this model to stacked ALMA data from \citet{Fujimoto19}. However, we are currently working on comparing the model with other observational probes from stacked \citep{ginolfi:2019} and individual \citep{Fujimoto:2020qzo} high-redshift systems. Moreover, new zoom-in cosmological simulations are available \citep{pallottini:2019}, and studying how the outputs of our model compare with these numerical results can give us some important insights on the numerical treatment of stellar feedback and on the role of catastrophic cooling in simulations. From these different comparisons, we also hope to explore in detail the capabilities of our model in constraining key properties of high redshift galaxies such as the mass outflow rate, the DM halo mass, and the escape fraction of ionizing photons. These are all crucial quantities that are hardly recovered from alternative methods at high redshifts. 

In spite of its success, the model developed here contains many simplifying assumptions, that limit its physical validity. For this reason, a different line of work we are exploring consists of an improvement of several hypotheses we made in \citet{Pizzati20} and of an analysis of how this affects our conclusions. 
Currently, we are trying to include the CMB suppression on the [CII] line emission \citep{dacunha2013, pallottini2017b, kohandel:2019}, which is particularly relevant in the external, low-density regions of the outflow.
Other important ingredients we want to include are non-equilibrium cooling/recombination, which is expected to have some physical effects when catastrophic cooling takes place \citep{gray2019catastrophic}, and the angular dependence of the escape fraction of ionizing photons, which depends on the disk-like morphology of the galaxy and would create a non-spherically symmetric outflow. 
Finally, in a longer-term plan we would work to develop a full three-dimensional numerical simulation of the outflow, thus dropping the steady-state assumption made here. A more realistic treatment of the circumgalactic environment would be possible in this context, and it would allow us to follow the formation of shocks and hydrodynamical instabilities in the interaction between the outflow and the external CGM/IGM. Although we do not expect these shocks to dramatically affect the derived overall outflow structure, the detailed profile and
extension of the [CII]-emitting region might turn out quantitatively different. Overall, we note how, despite the fact that some of these improvements might affect the quantitative conclusions of our model, it appears that so far outflows remain the best option to explain the nature of extended [CII] halos, and therefore further work is important to solve this puzzling problem in galaxy formation.


\section*{Note}

This document contains 24,847 characters (with spaces). The title, this note and the references are not included in the counting.


\bibliography{biblio}

\end{document}
